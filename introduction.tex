\section{Introduction}

Since space applications become more and more important for our everyday life, be it direct-broadcast services, environmental monitoring or even space travel, the understanding of the environment those applications operate in is crucial to ensure proper and smooth functioning.

To improve the understanding of the space environment and to examine different effects of the space environment on a spacecraft, this report will give an in-depth analysis of the Cluster-II Mission, a mission carried out by the European Space Agency (ESA) to study the interaction between the solar wind and the earth's magnetosphere and provide an unprecedented 3D-model of those interactions \citep{ESA:clusterWebsite}.

This report is focused especially on the space radiation environment of the Cluster-II FM8 (Tango) satellite, one of the four satellites of the Cluster-II mission, and on the radiation effects on solar arrays and different electronic devices.
For analysing the space environment a tool called SPENVIS (Space Environment Information System) developed by ESA is used.
SPENVIS is a Web-interface with a powerful backend that provides access to numerical calculations and evaluations of a user-defined orbit or mission. It also provides data on different effects as 
"[...] cosmic rays, natural radiation belts, solar energetic particles, plasmas, gases, and "micro-particles" [...]" \cite{ESA:spenvisWebsite}.

To provide an overview over the FM8 Tango Mission, the next chapter will introduce the mission objectives as well as the basic space environment this specific satellite faces. After, different radiation effects are examined and numerical calculations for solar arrays and memory devices are performed, followed by an analysis of their results. \citep{ESA:spenvisWebsite}.
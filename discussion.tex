\section{Conclusion}
In this report we analysed the Cluster-II FM-8 Tango Mission by the European Space Agency with regard to the space environment the satellite is in during its lifetime.

The first chapter introduced the mission of FM-8 Tango, where four similar satellites are building a very accurate 3D-Model of the earth's magentosphere to analyse and understand its regions better, and described the orbit and its different effects it has on the spacecraft, as drag, erosion and radiation effects.

Having defined these constraints, we modelled the mission with the tool SPENVIS, to run simulations on the spacraft environment, especially with focus on the radiation effects it has on the spacecraft.
Due to the degradation of the Azur 3G28 solar cells, which we assumed to be used on the spacecraft, as stated in the report instructions, we found out that the shielding thickness for the mission time should be 230 $\mu$m to reduce the incoming flux to have an acceptable degradation at the EOL.

In addition, we analysed the total ionising dose on a memory device to determine the necessary shielding to achieve a maximum
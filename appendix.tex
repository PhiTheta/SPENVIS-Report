%\setcounter{tocdepth}{1}
%
%\renewcommand{\thesection}{\Alph{subsection}}
%\setcounter{tocdepth}{0}
\section*{Appendix}
\addtocontents{toc}{\protect\setcounter{tocdepth}{1}}
%\appendix
\renewcommand{\thesection}{\Alph{subsection}}


\subsection{Orbital Parameters of Cluster-II FM8 Tango}
\label{tleParameters}
\texttt{CLUSTER II-FM8 (TANGO)\\
1 26464U 00045B 16087.81656212 .00000382 00000-0 00000+0 0 9996\\
2 26464 131.5572 328.3783 5181518 141.3516 0.4910 0.44219885 51441}

% data from 31. march
%CLUSTER II-FM8 (TANGO)  
%1 26464U 00045B   16092.33974625  .00000415  00000-0  00000+0 0  9997
%2 26464 131.5372 328.4372 5176277 141.4516   0.4873  0.44214714 51467

\vspace{1em}
%Table \ref{table:tle} shows the parameters extracted from the TLE data in a more readable format.

\begin{table}[h!]
\centering
\caption{Cluster-II Tango Parameters extracted of TLE set}
\label{table:tle}
\begin{tabular}{|c | c|} 
 \hline
 \textbf{Parameter} & \textbf{Value} \\ [0.5ex] 
 \hline
 Satellite Common Name &  CLUSTER II-FM8 TANGO\\ 
 Satellite Number &  26464\\
 Elset Classification &  U\\
 International Designator &  00\\
 Launch Number of the Year & 045\\
 Epoch Year & 16\\
 Epoch & 87.81656212\\
 BSTAR Drag Term &  0.00000382\\
 Inclination (deg) &  131.5572\\
 RAAN (deg) &  328.3782\\
 Eccentricity &  0.5181518\\
 Argument of Perigee (deg) &  141.3516\\
 Mean Anomaly (deg) &  0.4910\\
 Mean Motion (rev/day) &  0.44219885\\
 Rev number at epoch &  5144 \\ [1ex] 
 \hline
\end{tabular}
\end{table}
